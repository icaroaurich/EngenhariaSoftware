\documentclass{scrreprt}
\usepackage{underscore}
\usepackage{graphicx}
\usepackage[brazilian]{babel}
\usepackage[utf8]{inputenc}
\usepackage[T1]{fontenc}
\usepackage{tocbasic}
\usepackage{enumitem}
\usepackage{siunitx}



\usepackage[bookmarks=false]{hyperref}

\hypersetup{
    pdftitle={Software Requirement Specification},    % title
    pdfauthor={Jean-Philippe Eisenbarth},                     % author
    pdfsubject={TeX and LaTeX},                        % subject of the document
    pdfkeywords={TeX, LaTeX, graphics, images}, % list of keywords
    colorlinks=true,       % false: boxed links; true: colored links
    linkcolor=blue,       % color of internal links
    citecolor=black,       % color of links to bibliography
    filecolor=black,        % color of file links
    urlcolor=purple,        % color of external links
    linktoc=page            % only page is linked
}


\def\myversion{1.0 }
\date{}
\usepackage{hyperref}
\begin{document}

\begin{flushright}
    \rule{14cm}{5pt}\vskip1cm
    \begin{bfseries}
        \Huge{VISÃO DO PRODUTO}\\
        \vspace{4.5cm}
        \LARGE{Versão \myversion}\\
        \vspace{6cm}
        \today\\
    \end{bfseries}
\end{flushright}

\tableofcontents

\chapter{Introdução}

\section{Proposta}

O Documento de Visão visa definir a direção estratégica e os requisitos principais para o desenvolvimento de o Sistema de Controle de Portarias e Comissões. Este sistema será uma solução integrada para gerenciar eficientemente a criação, aprovação, monitoramento e arquivamento de portarias e comissões para o Instituto Federal da Bahia campus Eunápolis. A proposta é garantir um fluxo de trabalho automatizado e centralizado, melhorando a transparência, rastreabilidade e acessibilidade das informações.

\section{Finalidade}
A finalidade deste documento é fornecer uma base clara e compartilhada para todos os stakeholders envolvidos no desenvolvimento do sistema. Ele define os objetivos, os principais casos de uso, as necessidades dos usuários, e os requisitos funcionais e não funcionais do sistema. Com isso, busca-se alinhar as expectativas de todos os envolvidos, reduzir ambiguidades e orientar as fases subsequentes do projeto, desde o design até a implementação e testes.

\section{Scope}
O escopo deste documento abrange a definição das funcionalidades essenciais do Sistema de Controle de Portarias e Comissões. Inclui a gestão de todo o ciclo de vida das portarias e comissões, desde sua solicitação e elaboração, passando pela revisão e aprovação, até sua publicação e eventual revogação. O sistema também deverá permitir a gestão de participantes e responsáveis pelas comissões, controle de prazos, geração de relatórios, e notificações automatizadas.
\newline
O escopo não inclui integrações com sistemas externos fora da organização, mas poderá prever a exportação de dados em formatos padronizados. Também não abrange o desenvolvimento de módulos específicos para dispositivos móveis, embora a interface web seja projetada com responsividade em mente.
\chapter{Descrição Geral}

O \textbf{Sistema de Controle de Portarias e Comissões} é uma aplicação que será disponibilizada via rede e projetada para automatizar e centralizar a gestão de portarias e comissões do Instituto. O sistema tem como objetivo principal facilitar o processo de criação, revisão, aprovação e arquivamento de documentos oficiais relacionados a portarias e comissões, garantindo maior eficiência, transparência e segurança no gerenciamento dessas atividades, bem como o controle de pessoas envolvidas.

\section{Funcionalidades Principais}

\begin{itemize}
    \item \textbf{Gestão de Portarias}
    \begin{itemize}
        \item \textbf{Criação e Edição}: Permite que usuários autorizados criem e editem portarias, definindo seus termos, responsáveis, datas de vigência e outros parâmetros relevantes.
        \item \textbf{Fluxo de Aprovação}: Implementa um fluxo de trabalho para a revisão e aprovação das portarias, com notificações automáticas para os responsáveis em cada etapa.
        \item \textbf{Publicação e Arquivamento}: Após a aprovação, as portarias são publicadas e ficam acessíveis a todos os usuários autorizados, com um histórico completo de versões e alterações.
    \end{itemize}

    \item \textbf{Gestão de Comissões}
    \begin{itemize}
        \item \textbf{Formação e Designação}: Facilita a criação e designação de comissões, permitindo a inclusão de membros, definição de objetivos, prazos e responsabilidades.
        \item \textbf{Controle de Prazos}: Acompanhamento de prazos estabelecidos para as atividades das comissões, com alertas automáticos para os membros sobre tarefas e datas importantes.
        \item \textbf{Relatórios e Documentação}: Geração automática de relatórios sobre o progresso das comissões, reuniões realizadas, e decisões tomadas, com armazenamento centralizado dos documentos gerados.
    \end{itemize}

    \item \textbf{Notificações e Alertas}
    \begin{itemize}
        \item O sistema enviará notificações automáticas via e-mail ou diretamente na interface para alertar sobre prazos, solicitações de revisão, aprovações pendentes e outras atividades críticas.
    \end{itemize}

    \item \textbf{Busca e Acesso Facilitado}
    \begin{itemize}
        \item Ferramentas de busca avançada permitem que usuários encontrem portarias e comissões rapidamente por meio de filtros como datas, palavras-chave, responsáveis, status, entre outros critérios.
    \end{itemize}

    \item \textbf{Segurança e Controle de Acesso}
    \begin{itemize}
        \item O sistema implementa um controle rigoroso de acesso baseado em permissões, garantindo que apenas usuários autorizados possam criar, editar, ou visualizar certos documentos e informações sensíveis.
    \end{itemize}
\end{itemize}

\section{Arquitetura do Sistema}

O sistema será desenvolvido como uma aplicação web responsiva, acessível a partir de navegadores modernos. A arquitetura será baseada em uma estrutura cliente-servidor, onde o servidor central gerencia todas as operações de backend, incluindo banco de dados, lógica de negócios e autenticação de usuários. O frontend será projetado para ser intuitivo e fácil de usar, com uma interface limpa e acessível em diferentes dispositivos.

\section{Benefícios Esperados}

\begin{itemize}
    \item \textbf{Eficiência Operacional}: Redução do tempo e esforço necessários para gerenciar portarias e comissões, graças à automação e centralização dos processos.
    \item \textbf{Transparência e Rastreabilidade}: Acesso facilitado ao histórico de portarias e comissões, com rastreabilidade completa de revisões e aprovações.
    \item \textbf{Conformidade e Segurança}: Garantia de que todas as operações estão em conformidade com as políticas organizacionais, com segurança reforçada para documentos sensíveis.
\end{itemize}

Este sistema será uma ferramenta essencial para organizações que buscam melhorar o gerenciamento de seus processos administrativos, oferecendo um ambiente controlado e eficiente para a gestão de portarias e comissões.

\section{Ambientes Operacionais}
O sistema será acessivel para todos por qualquer sistema operacional: Baseados em Linux, Windows e Mac.

\chapter{Funções dos sistema}

\section{Módulos principais}
\begin{enumerate}
    \item Gerenciamento de pessoas:
    \item Gerenciamento de portarias:
    \item Gerenciamento de comissões:
    \item Exportação
\end{enumerate}


\section{Requisitos funcionais}


\begin{enumerate}[label=\textbf{RF-\arabic*}]
\item
O sistema deve ser publicado em servidor local e disponibilizado para acesso na intranet do IFBA;

\item
O sistema deve funcionar sem acesso à WAN, para as funcionalidades de gestão de pessoas, portarias e comissões mas deve funcionar em LAN (Intranet).
\end{enumerate}


\section{Requisitos não-funcionais}
\begin{enumerate}[label=\textbf{RF-\arabic*}]
\newcounter{RF}
\setcounter{RF}{1}
\item
O sistema deve ser capaz de gerenciar pessoas. A gestão de pessoas inclui cadastro, edição e exclusão. O cadastro das pessoas deve permitir a inserção dos seguintes dados: Nome, Matrícula (opcional); CPF; Área de atuação/Especialidade;

\item
O sistema deve possuir uma listagem de pessoas, a qual outras pessoas poderão  buscar e visualizar os dados completos, exceto os dados sensíveis. Deve ser incluido na visualização dos dados os vinculos com portarias e comissões;

\item
No gerenciamento de pessoas, o sistema deve lidar com dois tipos de perfis:
usuário e não usuário. Somente pessoas com perfil de usuário terá permissão para gerencias pessoas, portarias e permissões. Dessa forma, o sistema deve possuir um controle de acesso com usuário e senha;

\item
O sistema deve gerenciar portarias. A gestão de portarias inclui cadastro, edição e exclusão das mesmas. O cadastro das portarias deve permitir a inserção dos seguintes dados: Número, Data, Processo SEI (opcional), Resumo (motivação da portaria), Validade (opcional), Observação (informações pertinentes contidas ou não na portaria; opcional), Arquivo (opcional);

\item
As portarias podem possuir um relacionamento entre si. Por exemplo, uma portaria pode ser publicada para alterar o conteúdo de uma portaria anterior. Quando houver um relacionamento entre portarias, deve ser possível informar qual é o tipo de relacionamento: REVOGAÇÃO, RETIFICAÇÃO, COMPLEMENTAR, ALTERAÇÃO;

\item
As portarias cadastradas devem possuir um identificador único composto por um número sequencial e o ano de publicação. no fomato [número]/[ano];

\item
O sistema deve possuir uma listagem das portarias, a qual deve permitir a visualização; o ordenamento por:  data de publicação e identificador; e pesquisar por data, identificador e descrição;

\item
O sistema deve ser capaz de exportar as portarias cadastradas para o formato de CSV.

\end{enumerate}

\end{document}
